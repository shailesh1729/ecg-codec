%!TEX root = paper_ecg_cs_codec.tex
\section{Conclusion and Future Work}
\label{sec:conclusion}

\subsection{Contributions}

The key contributions of our encoder architecture are the following:

\begin{itemize}
    \item We have presented a digital compressive sensing based encoder
    design including adaptive digital quantization
    and entropy coding which can be
    implemented entirely using integer arithmetic.
    \item We have demonstrated that ANS based arithmetic coding can
    lead to significant space savings.
    \item We have demonstrated that we don't need a hard-coded
    code-book for entropy coding. We can simply model the
    measurement values distributed as quantized Gaussian values
    for effective entropy coding.
    \item We use simple mean and standard deviation as the entropy model
    parameters which can be dynamically adapted. The overhead of
    the side information is negligible.
    \item We have demonstrated up to $16\%$ of additional space
    savings contributed by the quantization and entropy coding steps.
    \item We have described a complete bitstream specification for
    carriage of all the side information needed to decode the
    entropy coded compressive measurements.
\end{itemize}

The software code implementing this codec
and all scripts for the experimental studies conducted
in this work have been released as open source software
on GitHub \cite{kumar2022ecgcodec}.

\subsection{Future Work}

The reconstruction algorithm in the decoder is replaceable.
Deep learning based architectures can provide further boost
to the quality of reconstruction. The impact of quantization
and clipping on deep learning based reconstruction needs
to be further studied.

The quantization, clipping and entropy coding blocks in our
codec design are general purpose and should be useful
in other CS based applications (e.g., EEG, EMG, FECG etc.).

During the course of long term ECG monitoring, one may
need to change the encoder parameters ($m,n,d$) to
respond to changing characteristics of the ECG data.
Enhancements in the bitstream specification can be
done accordingly.

One advantage of using fixed length coding for measurements
is random access to any part of the bitstream.
This is not straightforward in stream codes. ANS however
makes it easy to build a seakable decoder by introducing
checkpoints \cite{bamler2022constriction}. Such checkpoints
may be communicated as in the frame headers. 



