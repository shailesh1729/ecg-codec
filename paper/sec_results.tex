%!TEX root = paper_ecg_cs_codec.tex
\section{Results}
\label{sec:results}

\subsection{Impact of quantization parameter}
In \cref{fig:cs:codec:100:q:mr:stats}, we study
the impact of the quantization parameter $q$
on different compression statistics at different
values of measurement ratio $\text{mr}= \frac{m}{n}$
for record 100.
The number of measurements ($m$) have been chosen to vary from
$20\%$ to $60\%$ of the window size ($n$) in different configurations.
The quantization parameter varies from $0$ to $7$.
Nonadaptive quantization was used for this experiment.

(a) shows that reconstruction quality doesn't
change much from $q=0$ till $q=4$ after which it starts degrading.
As expected, the SNR/PRD degrades as $m$ is reduced.
At $30\%$ and below, the PRD is above $9\%$ which is unacceptable.

(b) shows that PSS (percentage space saving) increases linearly
with $q$. As $q$ increases, the size of the alphabet for
entropy coding reduces and this leads to increased space savings.

(c) shows the variation of the quality score with $q$ and we
can clearly see that quality score generally increases till
$q=4$ after which it starts decreasing.

The key result is shown in panel (d) which shows the variation
of PSS with PMS at different values of $q$.
PSS increases linearly with PMS. Also, PSS is much higher
than PMS at higher quantization levels.

We have also listed the numerical values in
\cref{tbl:cs:codec:100:pms:pss:q}.
The first column for $q=0$ depicts the case
where no quantization has been applied,
we are only doing clipping and entropy coding.
Since PSS is consistently higher than PMS,
it is clear that we need less than 11 bits per sample
on an average after entropy coding of the measurement
values. Increasing quantization linearly increases
the PSS. Since up to $q=4$, there is no noticeable
impact on PRD, as seen in panel (a), it is safe
to enjoy these savings in bit rate. 
At $40\%$ PMS, one can attain up to $69.3\%$ PSS without
losing any reconstruction quality while one can push
further up to $q=6$ to the PSS of $79.9\%$ with some
degradation in quality while still staying under
acceptable PRD. 
 


\begin{figure*}[htb]
    \centering % <-- added

  \subfloat[$q$ vs $\prd$]{%
  \includegraphics[width=0.32\linewidth]{images/rec_100_q_vs_prd_at_mr.pdf}}
    \hfill
  \subfloat[$q$ vs $\pss$]{%
  \includegraphics[width=0.32\linewidth]{images/rec_100_q_vs_pss_at_mr.pdf}}
    \hfill
  \subfloat[$q$ vs Quality Score]{%
  \includegraphics[width=0.32\linewidth]{images/rec_100_q_vs_qs_at_mr.pdf}}
    \\
  \subfloat[$\pms$ vs $\pss$]{%
  \includegraphics[width=0.32\linewidth]{images/rec_100_pms_vs_pss_at_q.pdf}}
    \hfill
  \subfloat[$q$ vs $\bps$]{%
  \includegraphics[width=0.32\linewidth]{images/rec_100_q_vs_bps_at_mr.pdf}}
    \hfill
  \subfloat[$q$ vs $\bpm$]{%
  \includegraphics[width=0.32\linewidth]{images/rec_100_q_vs_bpm_at_mr.pdf}}
  \caption{Variation of compression statistics vs quantization parameter
  at different measurement ratios $\frac{m}{n}$
  for record 100}
\label{fig:cs:codec:100:q:mr:stats}
\end{figure*}

\begin{table}[ht]
\tiny
\centering
\caption{$\pss$
at different values of $q$ and $\pms$ for record 100}
\begin{tabular}{lrrrrrrrr}
\toprule
q &    0 &    1 &    2 &    3 &    4 &    5 &    6 &    7 \\
PMS  &      &      &      &      &      &      &      &      \\
\midrule
40.0 & 47.5 & 53.0 & 58.4 & 63.8 & 69.3 & 74.6 & 79.9 & 84.8 \\
50.0 & 55.7 & 60.3 & 64.8 & 69.3 & 73.8 & 78.3 & 82.6 & 87.0 \\
55.0 & 59.8 & 63.9 & 67.9 & 72.0 & 76.1 & 80.1 & 84.0 & 88.0 \\
60.0 & 63.6 & 67.2 & 70.8 & 74.5 & 78.1 & 81.7 & 85.2 & 88.6 \\
65.0 & 68.0 & 71.1 & 74.3 & 77.5 & 80.6 & 83.8 & 86.8 & 89.7 \\
70.0 & 71.8 & 74.6 & 77.3 & 80.0 & 82.7 & 85.5 & 88.1 & 90.7 \\
75.0 & 76.5 & 78.7 & 81.0 & 83.2 & 85.5 & 87.7 & 90.0 & 92.1 \\
80.0 & 81.0 & 82.8 & 84.6 & 86.4 & 88.2 & 90.0 & 91.8 & 93.6 \\
\bottomrule
\end{tabular}
\label{tbl:cs:codec:100:pms:pss:q}
\end{table}



\subsection{Compression across different records}

\Cref{tbl:comp:stats:m=256:n=512:d=4}
shows the compression statistics for the
encoder configuration of $m=256,n=512,d=4$
and the decoder configuration of $b=32$ (block size).
The statistics are further summarized in
\cref{tbl:comp:summary:m=256:n=512:d=4}.
The column OH reports the overhead in percentage
of the bits needed to encode the stream header
and frame headers. The column Q\_SNR reports the
quantization noise introduced due to the quantization
and clipping steps in the encoder.
We can see that this configuration achieves
PSS between $66.05 \pm 2.78$\%.
We achieve an additional $16\%$ of space savings
over and above the PMS ($50\%$).
This additional space savings is the contribution
due to adaptive quantization and entropy coding.
The PRD is $5.6 \pm 1.8 \%$.
Out of the 48 records, only 4 records have a PRD
greater than 9\%. Maximum PRD is 10.2\%.
The quality score is $57.8 \pm 18.6$.
The header bits overhead is $0.21 \pm 0.01 \%$
which is negligible.
This configuration achieves of $3.7 \pm 0.2$ bits per sample.
Quantization SNR is $37.3 \pm 0.9$ dB.
Our adaptive quantization algorithm ensures that the
quantization SNR doesn't vary much.




