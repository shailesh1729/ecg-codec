%!TEX root = paper_ecg_cs_codec.tex
\section{ECG Database and Performance Metrics}
\label{sec:data}

We use the MIT-BIH Arrhythmia Database \cite{moody2001impact}
from PhysioNet \cite{goldberger2000physiobank}.
The database contains 48 half-hour excerpts of two-channel
ambulatory ECG recordings from 47 subjects.
The recordings were digitized at 360 samples per second
for both channels with 11-bit resolution over a 10mV range.
The samples can be read in both digital (integer) form or
as physical values (floating point) via the software provided
by PhysioNet.
We use the MLII signal (first channel)
from each recording in our experiments.

Each window of $n$
samples generates $m$ measurements by the
sensing equation $\by = \Phi \bx$.
Assume that we are encoding $s$ ECG samples where
$s = n w$ and $w$ is the number of signal windows
being encoded (across all frames).
Let the ECG signal be sampled by the ADC device
at a resolution of $r$ bits per sample
\footnote{For MIT-BIH Arrhythmia database, $r=11$.}.
Then the number of uncompressed bits is given by
$\bits_u = r s$.
Let the total number of compressed
bits corresponding to the $s$ ECG samples be
$\bits_c$ \footnote{This includes the overhead bits required
for the stream header and frame headers to be explained later.}.
Then the \emph{compression ratio} ($\compr$) is defined as
\begin{equation}
\compr \triangleq \frac{\bits_u}{\bits_c}.
\end{equation}
\emph{Percentage space saving} ($\pss$) is defined as
\begin{equation}
\pss \triangleq \frac{\bits_u - \bits_c}{\bits_u} \times 100.
\end{equation}
We call the ratio $m/n$ as the \emph{measurement ratio}.
We \emph{percentage measurement saving} ($\pms$) is defined as:
\begin{equation}
\pms \triangleq \frac{n - m}{n} \times 100.
\end{equation}
Another way of looking at the compressibility is how
many bits per sample ($\bps$) are needed on average in the compressed
bitstream. We define $\bps$ as:
\begin{equation}
\bps \triangleq \frac{\bits_c}{s}.
\end{equation}
Similarly, we can define \emph{bits per measurement} ($\bpm$) as:
\begin{equation}
\bpm \triangleq \frac{\bits_c}{m w}.
\end{equation}
The \emph{normalized root mean square error} is defined as
\begin{equation}
\nrmse (\bx, \tilde{\bx}) \triangleq \frac{\| \bx - \tilde{\bx}\|_2}{\| \bx \|_2}
\end{equation}
where $\bx$ is the original ECG signal and $\tilde{\bx}$
is the reconstructed signal.
A popular metric to measure the quality of reconstruction
of ECG signals is
\emph{percentage root mean square difference} ($\prd$):
\begin{equation}
\prd(\bx, \tilde{\bx}) \triangleq \nrmse(\bx, \tilde{\bx}) \times 100
\end{equation}
The \emph{signal to noise ratio} ($\snr$) is related to $\prd$ as
\begin{equation}
\snr \triangleq -20 \log_{10}(0.01 \prd).
\end{equation}
Zigel et al. \cite{zigel2000weighted} established 
a link between the diagnostic distortion and
the easy to measure $\prd$ metric.
\Cref{tbl-quality-prd-snr} shows the classified quality
and corresponding SNR and PRD ranges.
\begin{table}[ht]
\centering
\caption{Quality of Reconstruction \cite{zigel2000weighted}}
\begin{tabular}{lrr}
\toprule
Quality & PRD & SNR \\
\midrule 
Very good & $<$ 2\% & $>$ 33 dB \\
Good & 2-9\% & 20-33 dB \\
Undetermined & $\geq$ 9\% & $\leq$ 20 dB\\
\bottomrule
\end{tabular}
\label{tbl-quality-prd-snr}
\end{table}
