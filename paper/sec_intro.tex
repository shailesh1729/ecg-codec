%!TEX root = paper_ecg_cs_codec.tex
\section{Introduction}
\label{sec:intro}
Noncommunicable diseases (NCDs) account for 72\%
of all global deaths with cardiovascular diseases
(CVDs) accounting for 44\% of all NCD mortality
\cite{collins2019interact}.
For patients with CVDs, wearable device based remote ECG
monitoring plays a critical role in their disease
management and care. 

In wireless body area networks (WBAN)
based telemonitoring networks\cite{cao2009enabling},
the energy consumption on sensor nodes is
a primary design constraint \cite{milenkovic2006wireless}.
The wearable sensor nodes are often battery operated.
It is necessary to reduce energy consumption as
much as possible.

However, long term ECG monitoring
can generate a large amount of uncompressed data.
For example, each half hour 2 lead recording in the
MIT-BIH Arrhythmia database \cite{moody2001impact}
requires 1.9MB of storage. As shown in \cite{mamaghanian2011compressed},
in a real time telemonitoring sensor node, the wireless
transmission of data consumes most of the energy.
The real time compression of ECG data by a low
complexity encoder has received significant attention
in the past decade.

ECG signal compression has been an active area
of interest for several decades. Extensive surveys
can be found in \cite{singh2015review,rajankar2019electrocardiogram}.
Compressive sensing (CS) based techniques for ECG
data compression have been reviewed in \cite{craven2014compressed,kumar2022review}.

Transform domain techniques
(e.g., Discrete Cosine Transform \cite{al1995dynamic},
Discrete Cosine Transform \cite{batista2001compression,bendifallah2011improved},
Discrete Wavelet Transform \cite{djohan1995ecg,lu2000wavelet,pooyan2004wavelet,kim2006wavelet}) are popular in ECG compression
and achieve high compression ratios (CR) at clinically
acceptable quality.
However, these are computationally intensive sparsifying
transforms on all data samples and are thus not suitable
for WBAN sensor nodes \cite{craven2014compressed}.

Compressive sensing \cite{donoho2006compressed,baraniuk2007compressive,
candes2006compressive, candes2008introduction, candes2006near}
uses a sub-Nyquist sampling method by acquiring a small number
of incoherent measurements which are sufficient to reconstruct
the signal if the signal is sufficiently sparse in some
basis. For a sparse signal $\bx \in \RR^n$, one would make
$m$ linear measurements where $m \ll n$ which can be
mathematically represented by a sensing operation
\begin{equation}
\by = \Phi \bx
\end{equation}
where $\Phi \in \RR^{m \times n}$ is a matrix
representation of the sensing process and $\by \in \RR^m$
the set of $m$ measurements collected for $\bx$.
A suitable reconstruction algorithm can recover $\bx$
from $\by$.

Ideally, the sensing process should be implemented at the
hardware level in the analog to digital conversion (ADC) process.
However, much of the use of CS in ECG follows
a digital CS paradigm \cite{mamaghanian2011compressed} where
the ECG samples are acquired first by the ADC circuit on the
device and then they are translated into incoherent
measurements via the multiplication of a digital sensing matrix.
These measurements are then transmitted
to remote telemonitoring servers. A suitable reconstruction
algorithm is used on the server to recover the original
ECG signal from the compressive measurements.
Reconstruction algorithms for ECG signals include:
greedy algorithms 
\cite{polania2011compressed} (simultaneous orthogonal matching pursuit),
optimization based algorithms \cite{zhang2014energy},
\cite{mamaghanian2011compressed} (SPG-L1),
Bayesian learning based algorithms
\cite{zhang2012compressed,zhang2014spatiotemporal,zhang2013extension},
deep learning based algorithms \cite{zhang2021csnet}.

In order to keep the sensing matrix multiplication
simple and efficient, sparse binary sensing matrices
are a popular choice \cite{mamaghanian2011compressed,zhang2012compressed}.

\subsection{Problem Formulation}

We consider the problem of efficient transmission of
compressive measurements of ECG signals over the wireless body
area networks under the digital compressive sensing paradigm.
Let $\bx$ be an ECG signal and $\by$ be the corresponding
stream of compressive measurements. Our goal is to
transform $\by$ into a bitstream $\bs$ with as few bits
as possible without losing the signal reconstruction quality.

We consider whether a digital quantization of the compressive
measurements affects the reconstruction quality.
Further, we study the empirical distribution of compressive
measurements to explore efficient ways of entropy coding
of the quantized measurements.

A primary constraint in our design is that the encoder
should avoid any floating point arithmetic so that it
can be implemented efficiently in low power devices.


\subsection{Related Work}

The literature on the use of CS for ECG compression is mostly
focused on the design of the specific \emph{sensing matrix},
\emph{sparsifying dictionary}, or \emph{reconstruction algorithm}
for the high quality reconstruction of the ECG signal from the
compressive measurements.
To the best of our knowledge, (digital) quantization and entropy coding of
the compressive measurements of ECG data hasn't received
much attention in the past.

Mamaghanian et al.\cite{mamaghanian2011compressed} use a Huffman codebook
which is deployed inside the sensor device. They don't
use any quantization of the measurements.
However, they don't
provide much detail on how the codebook was designed or how should
it be adapted for variations in ECG signals.
They clearly define the compression ratio in terms
of a ratio between the uncompressed bits $\bits_u$
and the compressed bits $\bits_c$. They define it
as $\frac{\bits_u - \bits_c}{\bits_u} \times 100$.
This is often known in literature as
\emph{percentage space savings}.

Simulation based studies often send the real valued compressive
measurements to the decoder modules and don't consider the issue
of number of bits required to encode each measurement.
Zhang et al. \cite{zhang2012compressed},
Zhang et al. \cite{zhang2021csnet} define
$\frac{n - m}{n} \times 100$ as compression ratio. 
Mangia et al. \cite{mangia2020deep} define $\frac{n}{m}$
as the compression ratio.
Polania et al. \cite{polania2018compressed} define
$\frac{m}{n}$ as compression ratio.
Picariello et al. \cite{picariello2021novel} use
a non-random sensing matrix.
They infer a circulant binary sensing matrix
directly from the ECG signal being compressed.
The sensing matrix is adapted as and when
there is significant change in the signal
statistics.
They represent both the ECG samples and
compressive measurements with same number of
bits per sample/measurement. However,
in their case, there is the additional
overhead of sending the sensing matrix updates.
Their compression ratio is slightly lower than
$\frac{n}{m}$ where they call $\frac{n}{m}$ as
the \emph{under-sampling ratio}.


Huffman codes have frequently been used in ECG data
compression for non-CS methods.
Luo et al. in \cite{luo2014dynamic} proposed a dynamic
compression scheme for ECG signals which consisted
of a digital integrate and fire sampler followed
by an entropy coder. They used Huffman codes for
entropy coding of the timestamps associated with
the impulse train.
Chouakri et al. in \cite{chouakri2013wavelet} compute
the DWT of ECG signal by Db6 wavelet, select coefficients
using higher order statistics thresholding, then perform
linear prediction coding and Huffman coding of the selected
coefficients.  

 Asymmetric Numeral Systems (ANS) \cite{duda2013asymmetric}
 based entropy coding schemes have seen much success
 in recent years for lossless data compression.
 \emph{zstd} by Facebook Inc. \cite{zstd}
 and \emph{LZFSE} by Apple Inc. are popular lossless
 data compression formats based on ANS.
 To the best of our knowledge, ANS type stream codes
 have not been considered in the past for the entropy
 coding of compressive measurements.


\subsection{Contributions}
In this paper, we present an adaptive quantization and
entropy coding scheme for the digital compressive measurements
in the encoders running on the WBAN sensor nodes.
The adaptive quantization scheme has been designed in a manner
so that the quantization noise is kept within a specified
threshold. Once the measurements are quantized and clipped,
they are ready to be entropy coded. A quantized Gaussian
distribution of the measurements is estimated directly from
the data and the parameters for this distribution are used
for the entropy coding of the measurements.  
Asymmetric numeral systems (ANS) based
entropy coding is then used for efficient entropy coding
of the quantized and clipped measurements.
The ECG signal is split into frames for encoding purposes.
We have designed a bitstream format with a stream header
to describe the encoding parameters and a frame header
for each encoded frame containing frame wise quantization
parameters. We have designed a decoder which accepts this
encoded bitstream and reconstructs the ECG signal frame by
frame in real time. 

Our encoding scheme doesn't require a fixed codebook
for entropy coding. The encoder has been
designed in a manner so that all the operations can
be implemented by integer arithmetic. We have
tested our encoder on MIT-BIH Arrhythmia database
and are able to demonstrate additional space savings
beyond the savings from reduced number of measurements.
Our implementation is available as part of
reproducible research on GitHub \cite{kumar2022ecgcodec}.

\subsection{Paper Organization}
The rest of the paper is organized as follows.
\Cref{sec:data} describes the ECG database
used for the evaluation of the codec and the
performance metrics.
\Cref{sec:arch} describes the proposed codec architecture.
The analysis of the experimental results is covered in \cref{sec:results}.
\Cref{sec:conclusion} summarizes our contributions and major findings, and
identifies future work.
\Cref{appsec:cs} provides a short overview on
compressive sensing.
\Cref{appsec:ec} provides a short overview on
entropy coding.

