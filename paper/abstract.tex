%!TEX root = paper_ecg_cs_codec.tex
% As a general rule, do not put math, special symbols or citations
% in the abstract or keywords.
\begin{abstract}
Remote monitoring of electrocardiogram (ECG) signals
plays a critical role in the management of cardiovascular
diseases. Current long term ECG monitoring systems
generate a large amount of data that need to be
sent across a wireless body area network.
The wearable devices are resource limited.
Hence a low resource consuming compression
scheme is desirable.
Compressive sensing (CS) is quite appealing
as a low complexity method for compression of
ECG data. This paper investigates the distribution
of the compressive measurements and proposes
a CS based codec architecture comprising
simple quantization and
asymmetric numeral systems (ANS) based
entropy coding steps
that can significantly boost the compression ratio
without sacrificing reconstruction performance.
The encoder has low computational complexity
beyond the sensing operation
and can be implemented entirely using integer arithmetic.
The quantized Gaussian entropy model for the compressive measurements
is estimated directly from the data and can be easily adapted
to achieve better compression. Our decoder architecture
is flexible in choosing any suitable reconstruction method.
We have tested our encoder with bound optimization
based block sparse Bayesian learning (BSBL-BO)
reconstruction algorithm on MIT-BIH Arrhythmia database.
We have validated that our encoder can achieve
additional 20-30\% of space savings without any
impact on reconstruction quality on top of the
savings in terms of measurements. We have
also clearly described the digitization process
for the measurements which has often been ignored
in the literature on CS based compression of ECG data. 
As part of reproducible research we have open sourced
our codec (implemented in Python with JAX based GPU
acceleration for the decoder).
\end{abstract}
